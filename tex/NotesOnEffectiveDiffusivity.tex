\documentclass[11pt]{article}

%% WRY has commented out some unused packages %%
%% If needed, activate these by uncommenting
\usepackage{geometry}                % See geometry.pdf to learn the layout options. There are lots.
%\geometry{letterpaper}                   % ... or a4paper or a5paper or ... 
\geometry{a4paper,left=2.5cm,right=2.5cm,top=2.5cm,bottom=2.5cm}
%\geometry{landscape}                % Activate for rotated page geometry
%\usepackage[parfill]{parskip}    % Activate to begin paragraphs with an empty line rather than an indent

\usepackage{natbib}

%for figures
%\usepackage{graphicx}

\usepackage{color}
\definecolor{mygreen}{RGB}{28,172,0} % color values Red, Green, Blue
\definecolor{mylilas}{RGB}{170,55,241}
%% for graphics this one is also OK:
\usepackage{epsfig}

%% AMS mathsymbols are enabled with
\usepackage{amssymb,amsmath}

%% more options in enumerate
\usepackage{enumerate}
%\usepackage{enumitem}

%% insert code
\usepackage{listings}

\usepackage[utf8]{inputenc}

\usepackage{hyperref}

%% To make really wide whats that cover everything:
\usepackage{scalerel}
\usepackage{stackengine}
\stackMath
\def\hatgap{2pt}
\def\subdown{-2pt}
\newcommand\what[2][]{%
\renewcommand\stackalignment{l}%
\stackon[\hatgap]{#2}{%
\stretchto{%
    \scalerel*[\widthof{$#2$}]{\kern-.6pt\bigwedge\kern-.6pt}%
    {\rule[-\textheight/2]{1ex}{\textheight}}%WIDTH-LIMITED BIG WEDGE
}{0.5ex}% THIS SQUEEZES THE WEDGE TO 0.5ex HEIGHT
_{\smash{\belowbaseline[\subdown]{\scriptstyle#1}}}%
}}

% Default fixed font does not support bold face
\DeclareFixedFont{\ttb}{T1}{txtt}{bx}{n}{12} % for bold
\DeclareFixedFont{\ttm}{T1}{txtt}{m}{n}{12}  % for normal

% Custom colors
\usepackage{color}
\definecolor{deepblue}{rgb}{0,0,0.5}
\definecolor{deepred}{rgb}{0.6,0,0}
\definecolor{deepgreen}{rgb}{0,0.5,0}

% colors
\usepackage{graphicx,xcolor,lipsum}


\usepackage{mathtools}

\usepackage{graphicx}
\newcommand*{\matminus}{%
  \leavevmode
  \hphantom{0}%
  \llap{%
    \settowidth{\dimen0 }{$0$}%
    \resizebox{1.1\dimen0 }{\height}{$-$}%
  }%
}

% commmand for this doc
\newcommand{\So}{\mathcal{S}}
\newcommand{\bvth}{\bar{\vth}}
\newcommand{\kappat}{\kappa_{tot}}
\newcommand{\kappae}{\kappa_{e}}
\newcommand{\kappaN}{\kappa_{n}}
\newcommand{\kappaoc}{\kappa_{oc}}


\title{Notes on Ocean Eddy Diffusivity}
\author{C. B. Rocha\thanks{Scripps Institution of Oceanography, University
of California, San Diego. E-mail: crocha@ucsd.edu}\,\, \& W. R. Young}
\date{\today}

\begin{document}

\newcommand{\com}{\, ,}
\newcommand{\per}{\, .}

%% Averages
% Use \bar to over line solo symbols

\newcommand{\av}[1]{\bar{#1}}
\newcommand{\avbg}[1]{\overline{#1}}
\newcommand{\avbgg}[1]{\overline{#1}}

% A nice definition
\newcommand{\defn}{\ensuremath{\stackrel{\mathrm{def}}{=}}}

% space in equations
\newcommand{\qqand}{\qquad \text{and} \qquad}
\newcommand{\qand}{\quad \text{and} \quad}

% equations
\def\beq{\begin{equation}}
\def\eeq{\end{equation}}

\def\bea{\begin{align}}
\def\ena{\end{align}}


% calculus
\newcommand{\ord}{\mathcal{O}}
\newcommand{\p}{\partial}
\newcommand{\ii}{{\rm i}}
\newcommand{\dd}{{\rm d}}
\newcommand{\id}{{\, \rm d}}
\newcommand{\ee}{{\rm e}}
\newcommand{\DD}{{\rm D}}
\newcommand{\wavy}{\text{wavy}}
\newcommand{\qg}{\text{qg}}
\newcommand{\dt}{\Delta t}
\newcommand{\dx}{\Delta x}
\newcommand{\be}{\beta}

\newcommand{\al}{\alpha}
\newcommand{\bx}{\boldsymbol{x}}
\newcommand{\by}{\boldsymbol{y}}
\newcommand{\bu}{\boldsymbol{u}}
\newcommand{\bv}{\boldsymbol{v}}


\newcommand{\half}{\tfrac{1}{2}}
\newcommand{\halfrho}{\tfrac{1}{2}}
\newcommand{\rz}{{}}
\newcommand{\bn}{\boldsymbol{\hat n}}
\newcommand{\br}{\boldsymbol{r}}
\newcommand{\bR}{\boldsymbol{R}}
\newcommand{\bA}{\ensuremath {\boldsymbol {A}}}
\newcommand{\bB}{\ensuremath {\boldsymbol {B}}}
\newcommand{\bU}{\ensuremath {\boldsymbol {U}}}
\newcommand{\bE}{\ensuremath {\boldsymbol {E}}}
\newcommand{\bN}{\ensuremath {\boldsymbol {\mathrm{N}}}}
\newcommand{\bJ}{\ensuremath {\boldsymbol {J}}}
\newcommand{\bXX}{\ensuremath {\boldsymbol {\mathcal{X}}}}
\newcommand{\bFF}{\ensuremath {\boldsymbol {F}}}
\newcommand{\bF}{\ensuremath {\boldsymbol {F}^{\sharp}}}
\newcommand{\bG}{\ensuremath {\boldsymbol G}}
\newcommand{\bSigma}{\ensuremath {\boldsymbol {\Sigma}}}
\newcommand{\bvarphi}{\ensuremath {\boldsymbol {\varphi}}}
\newcommand{\bxi}{\ensuremath {\boldsymbol {\xi}}}
\newcommand{\avbxi}{\overline{\ensuremath {\boldsymbol {\xi}}}}

% math cal

\newcommand{\J}{\mathcal{J}}
\newcommand{\K}{\mathcal{K}}
\newcommand{\cG}{\mathcal{G}}
\newcommand{\cF}{\mathcal{F}}
\newcommand{\cN}{\mathcal{N}}
\newcommand{\cL}{\mathcal{L}}
\newcommand{\cS}{\mathcal{S}}
\newcommand{\cE}{\mathcal{E}}


% san serif for matrices and differential operators
%\newcommand{\helmn}{\mathsf{H}_n}
\newcommand{\helmm}{\triangle_m}
\newcommand{\helmn}{\triangle_n}
\newcommand{\helms}{\triangle_s}
\newcommand{\helm}{\triangle}
\newcommand{\sA}{\mathsf{A}}
\newcommand{\sB}{\mathsf{B}}
\newcommand{\sG}{\mathsf{G}}
\newcommand{\sI}{\mathsf{I}}
\newcommand{\sJ}{\mathsf{J}}
\newcommand{\gsJ}{\breve{\mathsf{J}}}
\newcommand{\sU}{\mathsf{U}}
\newcommand{\sP}{\mathsf{P}}
\newcommand{\sQ}{\mathsf{Q}}
\newcommand{\sR}{\mathsf{R}}
\newcommand{\sL}{\mathsf{L}}
\newcommand{\Lu}{\mathsf{L}(\what{u}_k)}
\newcommand{\Nu}{\mathsf{N}(\what{u}_k)}
\renewcommand{\L}{\mathsf{L}}
\newcommand{\N}{\mathsf{N}}
\newcommand{\sH}{\mathsf{H}}
\renewcommand{\sJ}{\mathsf{J}}
\renewcommand{\sI}{\mathsf{I}}
\renewcommand{\L}{\mathsf{L}}
\newcommand{\sM}{\mathsf{M}}
\newcommand{\sT}{\mathsf{T}}
\newcommand{\sGamma}{\mathsf{\Gamma}}
\newcommand{\sOmega}{\mathsf{\Omega}}
\newcommand{\sSigma}{\mathsf{\Omega}}
\newcommand{\sbeta}{\mathsf{\beta}}
\newcommand{\sPi}{\mathsf{\Pi}}
\newcommand{\sC}{\mathsf{C}}
\newcommand{\sQy}{\mathsf{Q}}
\renewcommand{\sb}{\mathsf{b}}

% u
\newcommand{\uhat}{\what{u}_k}

% angle brackets

\def\la{\langle}
\def\ra{\rangle}
\def\laa{\left \langle}
\def\raa{\right \rangle}


%grads and div's
\newcommand{\bcdot}{\hspace{-0.1em} \boldsymbol{\cdot} \hspace{-0.12em}}
\newcommand{\bnabla}{\boldsymbol{\nabla}}
\newcommand{\bnablaH}{\bnabla_{\! \mathrm{h}}}
\newcommand{\grad}{\bnabla}
\newcommand{\gradH}{\bnablaH}
\newcommand{\curl}{\bnabla \!\times\!}
\newcommand{\diver}{\bnabla \bcdot }
\newcommand{\cross}{\times}
%\newcommand{\lap}{\nabla^2}
\newcommand{\lap}{\triangle}

%varthetas and thetas
\newcommand{\vth}{\vartheta}
\newcommand{\psii}{\psi^{\mathrm{i}}}
\newcommand{\thb}{\theta^{\mathrm{-}}}
\newcommand{\vthb}{\vartheta^{\mathrm{-}}}
\newcommand{\vthbhat}{{\hat{\vartheta}}^{\mathrm{-}}}
\newcommand{\vThb}{\varTheta^{\mathrm{-}}}
\newcommand{\psib}{\psi^{\mathrm{-}}}
\newcommand{\tht}{\theta^{\mathrm{+}}}
\newcommand{\vtht}{\vartheta^{\mathrm{+}}}
\newcommand{\vththat}{{\hat{\vartheta}}^{\mathrm{+}}}
\newcommand{\vthtbhat}{{\hat{\vartheta}}^{\pm}}
\newcommand{\vTht}{\varTheta^{\mathrm{+}}}
\newcommand{\vthtb}{\vartheta^{\pm}}
\newcommand{\vThtb}{\varTheta^{\pm}}

% nondimensional numbers
\renewcommand{\Re}{\mathrm{Re}}
\newcommand{\Ro}{\mathrm{Ro}}
\newcommand{\Ri}{\mathrm{Ri}}

%psi's
%Galerking coefficient for psi:
\newcommand{\gpsi}{\breve \psi}
\newcommand{\gpsic}{{\breve \psi}^\star}
\newcommand{\gtau}{\breve \tau}
\newcommand{\gtauc}{{\breve \tau}^\star}
\newcommand{\gphi}{\breve \phi}
\newcommand{\gq}{\breve q}
\newcommand{\gU}{\breve U}
\newcommand{\gQ}{\breve Q}
\newcommand{\gsigma}{\breve \sigma}


\newcommand{\psit}{\psi^{\mathrm{+}}}
\newcommand{\psithat}{{\hat{\psi}}^{\mathrm{+}}}
\newcommand{\psibhat}{{\hat{\psi}}^{\mathrm{-}}}
\newcommand{\psitb}{\psi^{\pm}}
\newcommand{\psitbhat}{{\hat{\psi}}^\pm}
\newcommand{\St}{S^{\mathrm{+}}}
\newcommand{\Sb}{S^{\mathrm{-}}}
\newcommand{\phb}{\phi^{\mathrm{-}}}
\newcommand{\pht}{\phi^{\mathrm{+}}}
\newcommand{\tautb}{\tau^{\pm}}
\newcommand{\sigmatb}{\sigma^{\pm}}


\newcommand{\bur}{\left(\tfrac{f_0}{N}\right)^2}
\newcommand{\ibur}{\left(\tfrac{N}{f_0}\right)^2}
\newcommand{\Nm}{N_{\mathrm{mix}}}
\newcommand{\xim}{\xi_{\mathrm{mix}}}
\newcommand{\hs}{h_*}
\renewcommand{\sp}{\mathsf{p}}
\newcommand{\se}{\mathsf{e}}
\newcommand{\sptb}{\mathsf{p}^\pm}


%nmax is a problem:
%\newcommand{\nmax}{n_{\mathrm{max}}}
\newcommand{\nmax}{\mathrm{N}}
\newcommand{\mmax}{\mathrm{M}}

\newcommand{\WKB}{\mathrm{WKB}}
\newcommand{\Lam}{\Lambda}
\newcommand{\tha}{\theta}
\newcommand{\kap}{\kappa}
\newcommand{\bphi}{\boldsymbol{\phi}}
\newcommand{\third}{\tfrac{1}{3}}
\newcommand{\cs}{c^\star}
\newcommand{\dstar}{{\star\star}}
\newcommand{\nt}{n^{\mathrm{trnc}}}
\newcommand{\sDp}{\mathsf{D}^1_{\nmax}}
\newcommand{\sDpp}{\mathsf{D}^2_{\nmax}}
\newcommand{\sD}{\mathsf{D}}
\newcommand{\sDN}{\mathsf{D_\nmax}}
\newcommand{\sK}{\mathsf{K_2}}
\newcommand{\stheta}{\mathsf{\theta}}
\newcommand{\sphi}{\mathsf{\phi}}
\newcommand{\sq}{\mathsf{q}}
\newcommand{\cosech}{\text{csch}\,}
\newcommand{\sinc}{\text{sinc}\,}

%%%%%%%%% %%%%
\newcommand{\zp}{z^+}
\newcommand{\zm}{z^-}
\newcommand{\qA}{q^A_{\nmax}}
\newcommand{\psiB}{\psi^B_{\nmax}}
\newcommand{\phiB}{\phi^B_{\nmax}}
\newcommand{\eye}{\boldsymbol{\hat{i}}}
\newcommand{\jay}{\boldsymbol{\hat{j}}}
\newcommand{\kay}{\boldsymbol{\hat{k}}}
\newcommand{\psiG}{\psi^{\mathrm{G}}}
\newcommand{\qG}{q^{\mathrm{G}}}
\newcommand{\uG}{u^{\mathrm{G}}}
\newcommand{\UG}{U^{\mathrm{G}}}
\newcommand{\UGN}{U^{\mathrm{G}}_{\nmax}}
\newcommand{\QGN}{Q^{\mathrm{G}}_{\nmax}}
\newcommand{\sumoddn}{\sum_{n = 1, n~ \text{odd}}^{\nmax}}

% bretherton 
\newcommand{\qBr}{q_{\mathrm{Br}}}
\newcommand{\psiBr}{\psi_{\mathrm{Br}}}

\newcommand{\ep}{\epsilon}


\maketitle

\section{The Advection-Diffusion Equation}

\subsection{Preliminaries}
We are concerned with the lateral advection and diffusion of a  passive scalar (tracer). The evolution
of the tracer concentration $\vth$ is governed by
\beq
\label{ad}
\vth_t  + \vec{u}\cdot\nabla \vth =  \So + \kappa \lap \vth\com
\eeq
where $\So$ represents sources, $\kappa$ is the molecular diffusivity of the tracer $\vth$, and the horizontal laplacian
is $\lap \defn \p_x^2 + \p_y^2$. The advection-diffusion equation above is also often written as 
and \eqref{ad} can be re-written 
\beq
\label{adsf}
\vth_t  + \sJ(\psi,\vth) \vth =  \So + \kappa \lap \vth\com
\eeq
where the horizontal Jacobian is $\sJ(A,B) = A_x B_y - B_x A_y$, and the streamfunction $\psi$, associated with the non-divergent flow $\vec{u}$, is defined by
\beq
\label{sf}
u = -\psi_y\com \qquad \text{and} \qquad v = \psi_x\per
\eeq

\subsection{Zonnaly-averaged Equations}
For application in simple periodic domain, it is convenient to introduce zonally-averaged equations. In more complicated geometries, similar decompositions can be achieved given a well-defined average, which in practice is a combination of time and space averaging. Introducing the Reynolds decomposition
\begin{align}
\label{rey}
\vth(x,y,t) = \la\vth\ra_x(y,t)+ \vth'(x,y,t)\com\nonumber \\
\psi(x,y,t) = \la\psi\ra_x(y,t) + \psi'(x,y,t)\com \\
\end{align}
where
\beq
\la f \ra_x(y,t) = \frac{1}{L_x}\int_0^{L_x}\!\!\! f(x,y,t) \, \dd x\per
\eeq
With periodicity in $x$, the $x-$averaged $y-$velocity vanishes $\bar{v} = \bar{\psi}_x = 0$. 
The $x-$averaged tracer equation is then
\beq
\label{xave_ad}
\p_t \la\vth\ra_x + \p_y \la v' \vth'\ra_x = \la S \ra_x + \kappa \p_y^2 \la\vth\ra_x\per
\eeq
For completeness, the equation for the perturbation about the $x-$averaged tracer is
\beq
\label{pad}
\vth'_t + \p_x(u' \vth') + \p_y(v' \vth') + \p_x (u'\la\vth\ra_x) + \p_y(v' \la\vth\ra_x) - \p_y \la v'\vth'\ra_x = S' + \kappa (\p_y^2 \vth' + \p_x^2\vth)\per
\eeq

\subsection{Eddy diffusion}
This notes discusses different methods for estimating ocean eddy diffusivities. A common heuristic argument is parameterize eddy transport such as $\overline{v'\vth'}$ with a down-gradient
 eddy-difussivity
\beq
\label{eddy_visc}
\overline{v'\vth'} = -\kappae \p_y \bvth\per
\eeq
The introduction of the eddy diffusivity \eqref{eddy_visc} needs to be justified more formally. For now, it suffices to mention that  \eqref{eddy_visc} is a good approximation provided there is enough scale separation between the large-scale tracer gradient the eddy scales. Thus the parameterization \eqref{eddy_visc}  is used to close the zonally-averaged equation \eqref{xave_ad}:
\beq
\label{xave_ad_2}
\bvth_t   = \bar{S} + \kappat \p_y^2 \bvth\com
\eeq
where the total diffusivity is $\kappat \defn \kappae + \kappa$. There is significant inconsistency in the literature regarding 
notation and naming of diffusivities.  The term ``effective diffusivity''  is used to refer to $\kappat$ by some authors. Other investigators refer to $\kappae$ as the ``effective diffusivity'' likely recognizing that for large Peclet number $\kappat \approx \kappae$. The term ``eddy diffusivity'' is equally used to refer to both $\kappat$ or $\kappae$.

\subsection{Integral variance budget}
With harmless boundary conditions (e.g., double periodicity or no-flux across the boundaries) the tracer variance
equation is
\beq
\label{varvth}
\frac{\dd}{\dd t}  \int \half \vth^2 \dd A = \int \vth \So \dd A - \kappa \int | \nabla \vth |^2 \dd A\per
\eeq
       

%We can also write equations for the tracer variance $\bvth^2$
%\beq
%\label{varbvth}
%\frac{\dd}{\dd t}  \int \half \bvth^2 \dd y = \int \overline{v'\vth'} \bvth_y \dd y + \int \bvth \bar{\So} \dd y - \kappa\!\! \int (\p_y \bvth)^2 \dd y\com
%\eeq
%and the tracer variance $\bar{\vth'}^2$
%\beq
%\label{varpvth}
%\frac{\dd}{\dd t}  \int \half \vth'^2 \dd A =  -\int \overline{v'\vth'} \bvth_y \dd y   + \int \vth' \So' \dd A - \kappa \int \left[ |\nabla \vth' |^2 + \p_x \vth' \p_x \bvth \right] \dd A \per
%\eeq


\section{Renovated Waves on a Lattice}

To begin exploring the accuracy of different methods to estimate ocean eddy diffusivities,
we use a simple advection-diffusion model on a lattice first proposed by Pierrehumbert (2000).
The idea is to break the advection and difussion in different steps. The advection step is further
separated in two sub-steps: advection in the x-direction and y-direction. The advection sub-steps are
performed on a lattice. That is, the advection in the x-direction corresponds to a shift in the 
x-direction, and the advection in the y-direction corresponds to a shift in the y-direction:
\beq
i_x^{n+1} = i_x^{n} - \text{int}\left[u(y) \tau/2 \right]\com
\eeq
and
\beq
i_y^{n+1} = i_y^{n} - \text{int}\left[v(x) \tau/2 \right]\com
\eeq
where the superscripts represent the iteration and $\tau$ is the length of the renovation cycle, typically a eddy-turnover timescale. Figure \ref{latticei} illustrates the advection sub-steps in a couple of renovation cycles.
Clearly, a very complex tracer pattern emerges after a couple of renovation cycles. Notice that this advection scheme exactly conserves the probability density of the tracer if no diffusion is applied (the lattice is just re-combined).

In a slight generalization of Pierrehumbert's model, we represent the non-divergent velocity field as linear combination of a spectrum of waves with random phase:
\beq
\label{un}
u(y) = C \sum_{j=j_{min}}^{j_{max}} \left(\frac{j}{j_{min}}\right)^{-p/2}\!\!\! \cos\left(\frac{2\pi y}{L_y} j + \phi_j\right)\com
\eeq
\beq
\label{vn}
v(x) = C \sum_{j=j_{min}}^{j_{max}} \left(\frac{j}{j_{min}}\right)^{-p/2}\!\!\! \cos\left(\frac{2\pi x}{L_x} j + \psi_j\right)\com
\eeq
where $L_x$ and $L_y$ are the dimensions of the periodic domain, $\psi_n$ and $\phi_n$ are random phases drawn from a uniform distribution on $[0, 2\pi]$. Because these random phases are changed every iteration, this simple velocity field is termed renovated wave model. Note that in this generalized renovated wave model the kinetic energy spectrum of the flow follows a $j^{-p}$ power-law. Also in \eqref{un} and \eqref{vn}, $C$ is a normalization constant, determined so that the root-mean square velocity is prescribed 
\beq
C = 2 u_{rms}\left(\sum_{j=j_{min}}^{j_{max}} \left(\frac{j}{j_{min}}\right)^{-p}\right)^{-1/2}\per
\eeq

\begin{figure}[ht]
    \centering
       \includegraphics[width=0.3\textwidth]{figs/AdvExample1.png}
       \includegraphics[width=0.3\textwidth]{figs/AdvExample2.png}
       \includegraphics[width=0.3\textwidth]{figs/AdvExample3.png}\\
       \includegraphics[width=0.3\textwidth]{figs/AdvExample4.png}
       \includegraphics[width=0.3\textwidth]{figs/AdvExample5.png}
       \includegraphics[width=0.3\textwidth]{figs/AdvExample6.png}\\
       \includegraphics[width=0.3\textwidth]{figs/AdvExample7.png}
       \includegraphics[width=0.3\textwidth]{figs/AdvExample8.png}
       \includegraphics[width=0.3\textwidth]{figs/AdvExample9.png}
       \caption{\small The advection on the two-dimensional periodic lattice. Notice that a complex pattern starts to emerge
    		after a couple of renovation cycles. With a bigger lattice, one no longer sees the pixel fragmentation,
		and the differential advection results in strong filamentation.}
		    \label{latticei}
\end{figure}

Pierrehumbert performs the diffusion step in physical space, as a simple average of neighboring points, corresponding
to a second-order finite-difference approximation for the laplacian operator. Here we take advantage of the periodicity
of this model, and perform the diffusion step more accurately in Fourier space:
\beq
\label{diff}
\hat{\vth}^{n+1/2}_{k,l} = \hat{\vth}^{n}_{k,l} \ee^{-\kappa(k^2+l^2)\tau/2}\per
\eeq

\subsection{Advection-diffusion with constant background tracer gradient}
Instead of using a large-scale source, we implement a constant background gradient $G$,
so that the tracer concentration equation is
\beq
\vth_t  + \vec{u}\cdot\nabla \vth + G v =  \kappa \lap \vth\per
\eeq

The $x-$averaged variance budget is
\beq
\label{xave_budget_Gy}
\p_ t \la \half \vth \raxt +  \p_y \la \half v \vth^2 \raxt + G \la v \vth \raxt  = \p_y^2 \la \half \vth^2 \raxt -\kappa \la  |\nabla \vth|^2\raxt \per
\eeq
Because the background gradient is constant, the system is statistically homogeneous in $y$. Thus, in statistical steady state,
the $x-$averaged variance budget \eqref{xave_budget_Gy} is exactly given by a balance between variance production and dissipation:
\beq
G \la v \vth \raxt  = -\kappa \la  |\nabla \vth|^2\raxt \per
\eeq

\subsection{Spectral characteristics of RW simulations}
Our main interest is to use the lattice model to understand the potential biases and limitations of different methods
to estimate eddy diffusivities, with particular attention to the dependence on the kinetic energy spectral slope. While the
RW is simple enough to allow us to calculate its diffusivity exactly, its simulations display many characteristic of stirring
and mixing by turbulent flows. In particular, the tracer dynamics dependence on the characteristics of the flow is well-represented in the RW model. Figure \ref{Spectra8192} shows spectra from two $8192\times8192$ RW simulations.
With steep kinetic energy spectrum $k^{-4}$ (c), the tracer stirring is spectrally non-local (dominated by the energy containing eddies $k_{min} = 1$). The tracer variance spectrum is very shallow, approximately following a $k^{-1}$ power law across all
scales. In contrast, with shallow kinetic energy spectrum $k^{-2}$ (a), the tracer stirring is spectrally local, and the tracer variance spectrum approximately follow the same power law as the kinetic energy spectrum $k^{-2}$. At about the maximum wavenumber of the flow (the smallest stirring scale; $k_{max}=150$), the tracer variance spectrum transitions to $k^{-1}$,
Batchelor spectrum (see Vallis 2006; Scott 2006).

\begin{figure}[ht]
    \centering
       \includegraphics[width=0.45\textwidth]{figs/wavenumber_ke_spectrum_p_2.pdf}
       \includegraphics[width=0.45\textwidth]{figs/wavenumber_ke_spectrum_p_4.pdf}\\
       \includegraphics[width=0.45\textwidth]{figs/wavenumber_tracer_spectrum_p_2.pdf}
       \includegraphics[width=0.45\textwidth]{figs/wavenumber_tracer_spectrum_p_4.pdf}
       \caption{\small Kinetic energy spectra (a and c) and tracer variance spectra (b and d)  from RW simulations with different
       				spectral slopes of the kinetic energy. The simulations were run with same level of energy ($u_{rms}^2/2$)
				on a $8192\times8192$ lattice. In this experiments, the minimum stirring wavenumber is $k_{min}=1$ and
				the maximum stirring wavenumber is $k_{max}=150$.}
		    \label{Spectra8192}
\end{figure}


\subsection{The Osborn-Cox method}
 
Cox and Osborn introduced an eddy diffusion closure for the tracer flux,
\beq
\label{oc_diff}
\la v \vth \raxt = - \kappaoc G\com
\eeq
to obtain an explicit expression for the eddy diffusivity
\beq
\label{koc}
\kappaoc(y) = \frac{\la |\nabla \vth|^2 \raxt}{G^2} \kappa\per
\eeq
Note that the Osborn-Cox eddy diffusivity is simply an amplification of the molecular diffusivity, with the amplification
factor
\beq
A_{oc} = \frac{\la |\nabla \vth|^2 \raxt}{G^2}\per
\eeq
The amplification factor can be thought as a ratio of two length scales. In particular the perturbation of tracer scale as
$\vth \sim l_{mix} G$, where $l_{mix}$ is a mixing-length scale. Gradients of $\vth$ occur on relatively small scales extending
down to the Batchelor scale,
\beq
l_b = \left(\frac{\kappa}{S}\right)^{1/2}\com
\eeq
where $S$ is a characteristic scale of the rate of strain of the eddy field. Hence,
\beq
\label{Aoc_scaling}
A_{oc} \sim \frac{l_{mix}^2}{l_{b}^2} \Longrightarrow \kappaoc \sim l_{mix}^2 S\per
\eeq
It is pleasing that the eddy diffusivity scaling \eqref{Aoc_scaling} is independent of the molecular diffusivity $\kappa$.

\subsection{The Nakamura method}

A different approach to estimating the eddy diffusivity is the method of Nakamura ). In particular, Nakamura showed that averaging the advection-diffusion equation \eqref{ad} along tracer contours eliminates the advective eddy transport. The transformed equation resemble a diffusion equation in area coordinate with diffusivity (see appendix A for a derivation of the method):
\beq
\label{kN}
K_{n}(\Th, t) = L_{eq}^2 \kappa\com
\eeq
where $L_{eq}$ is the equivalent length of the contour. $K_N$ is an instantaneous measure of mixing as a function of each contour $\Th$; $K_n$it has units of (length)$^4$ (time)$^{-1}$. The Nakamura eddy diffusivity is then defined as
\beq
\kappaN(\Th) = \la K_n\ra_t/L_{min}^2\com
\eeq
where $L_{min}$ is the minimum length of the contour ($L_x$ in this case). One generally maps the Nakamura diffusivity from 
 contour $\Th$ to physical space $\Th = \Th(y)$ either using the structure of the background tracer concentration ($G y$) or by constructing a simple one-to-one map based on the distribution of the stirred tracer field.

\subsection{RW model simulations}
We now diagnose eddy diffusivities using both Osborn-Cox and Nakamura methods in reference simulations of the Renovated Waves (RW) model.  The RW model implemented with background constant tracer concentration $G$ is a plain-vanilla example of ``Statistically Homogeneous Isotropic Transport'' setup where the eddy diffusivity is given by Einstein's formula
\beq
\label{einstein_eddy_diffusivity}
\kappa_{ein} = \frac{\la (\Delta x)^2\ra}{2\tau}\per
\eeq 
In the RW model we have
\beq
\label{dxdt}
\dot x = u(y)\com
\eeq
with $u(y)$ given by the series \eqref{un}. Hence
\beq
\label{dx2}
\la (\Delta x)^2 \ra = \la (x - x_0)^2 \ra = u_{rms}^2 \tau^2 \la \cos (l y + \psi_j) \ra_{\psi} = \frac{u_{rms}^2 \tau^2}{2}\per 
\eeq
Thus
\beq
\label{einstein_eddy_diffusivity}
\kappa_{ein} = \frac{u_{rms}^2 \tau}{4}\per
\eeq 
Of course, in this case, the isotropic velocity field implies $\la (\Delta y)^2\ra = \la (\Delta y)^2\ra$ and the diffusivity is independent of direction.

Another important property of this model with constant background tracer gradient is that the scale separation between the mean tracer concentration and perturbations is infinite, and therefore the eddy diffusion closure is well justified.

\subsection{Some results}

Here we present results from two simulations akin of those figure \ref{Spectra8192} but with a little coarser resolution ($4096\times 4096$). Details in table \ref{RWExp}.
\begin{table}
\begin{center}
   \caption{\small Parameter of the RW experiments on a lattice.}
    \begin{tabular}{  l | l}
    \hline
    \hline
    	Size of the lattice & $n_x = N_y = 4096$\\
	Domain size & $L_x = L_y = 2 \pi$\\
	Velocity root-mean-square & $u_{rms} = 1$ \\
	Small scale diffusivity & $\kappa = 10^{-4} $\\
	Eddy decorrelation time scale & $\tau = 0.5$ \\
	Peclet number & \textsf{Pe} $= 5 \times10^3 $\\
    \hline
    \end{tabular}
    \end{center}
    \label{RWExp}
        \end{table}



Figure \eqref{Resultsp2Variance} shows  variance time series for two experiments with the parameters described above but with distinct kinetic energy spectrum: $p=2$ and $p=4$. Both experiments achieve statistical steady state  no later than about 20 renovation cycles --- the diagnostics discussed below are calculated after equilibration. Figure \ref{Resultsp4} shows a snapshot of tracer concentration together with the probability density function of the tracer concentration. Also shown in figure
  \ref{Resultsp4} are the kinetic energy spectrum and the tracer spectrum ? the latter exhibits an approximate $k^{-1}$ power law characteristics of non-local mixing. Results for the $p=2$ experiments display different properties: the stirring is is local in the range of active stirring and becomes non-local at wavenumbers larger than $k_{max}$ (see the transition in the tracer spectrum in figure \ref{Resultsp2}). The tracer concentration displays much rougher contours and its probability density function is less Gaussian than the one for the $p=4$ experiment.

\begin{figure}[ht]
    \centering
            \includegraphics[width=.9\textwidth]{figs/variance_time_series_2_2048.pdf}\\
                        \includegraphics[width=.9\textwidth]{figs/variance_time_series_4_4096.pdf}\\
       \caption{\small Results for RW simulation  Results $p=2$ and $p=4$.}
		    \label{Resultsp2Variance}
\end{figure}

\begin{figure}[ht]
    \centering
       \includegraphics[width=0.45\textwidth]{figs/tracer_snapshot_p_4_4096.pdf}
       \includegraphics[width=0.45\textwidth]{figs/wavenumber_ke_spectrum_p_4_4096.pdf}\\
         \includegraphics[width=0.45\textwidth]{figs/pdf_p_4_4096.pdf}
       \includegraphics[width=0.45\textwidth]{figs/wavenumber_tracer_spectrum_p_4_4096.pdf}
       \caption{\small Results for $p=4$ experiment. Snapshot of tracer concentration (top left)  and associated
       			probability density function (lower left). Kinetic energy spectrum (top right) and 
			tracer variance spectrum (lower right).}
		    \label{Resultsp4}
\end{figure}



\begin{figure}[ht]
    \centering
       \includegraphics[width=0.45\textwidth]{figs/tracer_snapshot_p_2_2048.pdf}
       \includegraphics[width=0.45\textwidth]{figs/wavenumber_ke_spectrum_p_2_2048.pdf}\\
         \includegraphics[width=0.45\textwidth]{figs/pdf_p_2_2048.pdf}
       \includegraphics[width=0.45\textwidth]{figs/wavenumber_tracer_spectrum_p_2_2048.pdf}
       \caption{\small Results for $p=2$ experiment: Snapshot of tracer concentration (top left)  and associated
       			probability density function (lower left). Kinetic energy spectrum (top right) and 
			tracer variance spectrum (lower right).}
		    \label{Resultsp2}
\end{figure}

Figure \eqref{VarianceBudgetp4} shows the $x-$averaged tracer variance budget and the estimates of eddy diffusivity.
The tracer budget is nearly closed (innacuracies are associated with the diffusion pulsing) and it is dominated by the 
balance between nearly $y-$independent production and dissipation, which are positive and negative definite, respectively. The other terms are smaller and integrate to zero. The estimates of eddy diffusivity are also essentially independent of $y$. The $y-$averaged value of both diffusivities calculated using Osborn-Cox and Nakamura method is about $1\%$ accurate (more averaging and substeps within the renovation cycles could further reduce this inacuracy).

\begin{figure}[ht]
    \centering
       \includegraphics[width=0.45\textwidth]{figs/tracer_budget_p_4_4096.pdf}
       \includegraphics[width=0.45\textwidth]{figs/diffusivity_p_4_4096.pdf}
       \caption{\small Results for $p=4$ experiment: Tracer variance budget (left) and estimates of eddy diffusivity (right).}
		    \label{VarianceBudgetp4}
\end{figure}

\clearpage
% large scale source
\subsection{Advection-diffusion with a simple large-scale source}

We employ a simple large-scale source
\beq
S(y) = \cos \left(\frac{2 \pi}{L_y}y\right)\per
\eeq
Because the source is only a function of $y$, half of the source is performed after the $x-$advection
substep
\beq
\label{source_x}
\vth^{n+1/2} = \vth^{n+1/2} + \frac{\dt}{2} S(y)\per
\eeq
The other half of the source could be applied in the same fashion, after the $y-$advection
substep
\beq
\label{source_y}
\vth^{n+1} = \vth^{n+1} + \frac{\dt}{2} S(y)\per
\eeq
In principle \eqref{source_y} is a brutal way to apply the source since source term is not invariant in $y$.
To apply the source more accurately, we note that in the ``$y$-advection''+``half-source'' substeps, we are
solving
\beq
\vth_t + v(x) \vth_y = \cos \left(\frac{2 \pi}{L_y}y\right)\com
\eeq
which can be easily integrated along the characteristics $y = y_0 + v \dt/2$, to give
\beq
\label{adv_source_y}
\vth(x,y,t+\dt) = \vth(x,y,\dt) + \frac{\sin\left[ k_1 (y_0 + v \dt/2 )\right] - \sin k_1 y_0}{k_1 v}\per
\eeq
where $k_1 \defn \frac{2\pi}{L_y}$. We anticipate loss of accuracy for in regions where $v\approx 0$. To
avoid this inaccuracies, we identify these points numerically, and use the exact value in limit $\dt \to 0$.
This limit can be calculated using the L'Hospital rule, simply expanding about $v=0$, to obtain
\beq
\frac{\sin\left[ k_1 (y_0 + v \dt/2 )\right] - \sin k_1 y_0}{k_1 v} \to \frac{\dt}{2}\cos k_1 y_0 \qquad \text{as}
\qquad v \to 0\per
\eeq

%In summary, the algorithm for this forced advection-diffusion problem on a lattice, every iteration is
%composed by the following substeps:
%\begin{enumerate}
%    \item $x-$advection
%    \item half source
%    \item $y-$advection + half source
%    \item diffusion
%\end{enumerate}
%Of course, one could break the diffusion step in two substeps, to be performed after the advection.
%In practice, there is no significant difference, and we opt to use the single step difusion above
%for computational efficiency. 




\subsection{Mean tracer concentration}
The $x-$averaged equation tracer equation is
\beq
\label{xave_cos}
\la\vth\ra_x = \cos k_1 y + (\kappa_e+\kappa) \p_y^2 \la\vth\ra_x\com
\eeq
where we used an eddy diffusivity to parameterize the tracer flux with the eddy diffusivity $\kappa_e >>\kappa$. In statistical steady state, averaging either in time or over ensembles, we obtain, to $\mathcal{O}(\kappa/\kappa_e)$,
\beq
\kappa_e \p_y^2\la \vth\ra_{x,t} = -\cos k_1 y\com
\eeq
Thus,
\beq
\label{pred_vthb}
\langle \bvth \rangle = \frac{\cos k_1 y}{\kappa_e k_1^2}\per
\eeq
To obtain a prediction for the mean concentration, we use the the exact eddy diffusivity, given by the Einstein's formula:
\beq
\label{einstein_diff}
\kappa_{ein} = \frac{\langle (\dx)^2 \rangle}{2 \tau}=\frac{u_{rms}^2 \tau}{4}\per
\eeq


Figure \ref{p35} shows a comparison between the prediction given by \eqref{pred_vthb} and numerical results from RW simulations. With enough scale separation between the large scale source and the eddies, there
is spectacular agreement between theory and numerics, independent of the slope of the kinetic energy spectrum (see figure \ref{p35} top). Even without scale separation (i.e., when eddy diffusivity is not formally justified),
 the time average of the $x-$averaged tracer concentration is reasonably consistent with the theoretical prediction. However, the are visible discrepancies between snapshots of $x-$averaged tracer concentration and the theoretical prediction; the discrepancies between snapshot and theory are larger with steep kinetic energy spectral slope since there is less self-averaging performed by the $x-$averaged operation.


\begin{figure}[ht]
    \centering
    \includegraphics[width=.45\textwidth]{figs/meansource_p_4_nmin_5.pdf}
     \includegraphics[width=.45\textwidth]{figs/meansource_p_2_nmin_5.pdf}\\
    \includegraphics[width=.45\textwidth]{figs/meansource_p_4_nmin_1.pdf}
     \includegraphics[width=.45\textwidth]{figs/meansource_p_2_nmin_1.pdf}\\
    \caption{$x-$averaged tracer concentration simulated in $512\times 512$ RW simulation with $\kappa = 5 \times 10^{-4}$. 
        with $j_{min}=5$ (top) and $j_{min}=1$(bottom).}
            \label{p35}
\end{figure}

\
\section{Diffusion suppression by a mean flow}
To study the suppression of eddy diffusivity by a mean flow, we simply add a constant zonal velocity to 
\eqref{un} and \eqref{vn}. The $x-$advection on the lattice is only slightly changed:

\begin{align}
\dot{x} &= U + u(y)\com  [0,\tau/2] :  x= x_0 + [U + u(y)]\frac{\tau}{2}\per \nonumber \\
\dot{y} &= 0\com   [0,\tau/2]  :             y = y_0\per
\end{align}
The $y$-advection is
\begin{align}
\dot{x} &= U  \com  [\tau/2,\tau] :  x= x_0 + U\tau/2\per \nonumber \\
\dot{y} &= v(x)\com  [\tau/2,\tau]  :            \dot{y} =  v(x_0  + U \tau/2)\per
\end{align}
Hence
\beq
y = y_0 + C \sum_{j=j_{min}}^{j_{max}} \left(\frac{j}{j_{min}}\right)^{-p/2}\!\!\! \frac{1}{U k} \left[\sin\left(k (x + U \tau) j + \psi_j\right)-\sin\left(k (x + U \tau/2) j + \psi_j \right)\right]\com
\eeq
The mean square displacement is
\beq
\la  (\Delta y )^2\ra =  C^2 \sum_j \left( \frac{j}{j_{min}} \right)^{-p} \left[\frac{1- \cos\left(U k \tau/2\right)}{(Uk)^2}\right]\per
\eeq
Note that in the limit of small mean velocity, $Uk \tau/2 \to 0$, we obtain the mean-square displacement for the non-mean-flow case:
\beq
\la  (\Delta y )^2\ra = u_{rms}^2 \frac{\tau^2}{2}\per
\eeq
We can then define the nondimensional parameter
\beq
\label{betah}
\beta(U,p,j_{min},j_{max}) = \frac{C^2 \sum_j \left(\frac{j}{j_{min}}\right)^{-p}\left(\frac{1-\cos(U k \tau/2)}{(U k)^2}\right)}{u_{rms}^2 \frac{\tau^2}{2}} =  \frac{2  \sum_j \left(\frac{j}{j_{min}}\right)^{-p}\left(\frac{1-\cos(U k \tau/2)}{(U k \tau/2)^2}\right)}{\sum_j \left(\frac{j}{j_{min}}\right)^{-p}} \com
\eeq
which is a measure of suppression (no suppression with $\beta=1$, full suppression with $\beta=0$).  Figure \ref{Suppresion}
shows $\beta$ as a function of the mean flow speed $U$ and the kinetic energy spectrum exponent  $p$ for different maximum
wavenumber $j_{max}$. Flows with shallow kinetic energy spectrum are significantly suppressed by the mean flow. This is 
because for these flows, the small scales dominate the stirring, but the small scale eddies are significantly sweep by the mean flow. In this local stirring regime, the relevant bulk non-dimensional number is
\beq
\label{ukmax}
U k_{max} \tau/2\com
\eeq
which can large even when $U$ is very small. On the other hand, flows with steep kinetic energy spectrum, the stirring is non-local and the relevant bulk non-dimensional parameter is
\beq
\label{ukmax}
U k_{min} \tau/2\com
\eeq
 and the eddy turn-over time scale of the energy containing eddies relatively large compared to the mean advective time-scale, so that a significantly large mean flow is necessary to suppress the eddy diffusivity.
 
 Figure \ref{Suppresion2} shows the $\beta$ as a function of wavenumber and mean flow speed, that is
 \beq
\label{betak}
\beta(U,k) =   \frac{2  \sum_{j} \left(\frac{j}{j_{min}}\right)^{-p}\left(\frac{1-\cos(U k \tau/2)}{(U k \tau/2)^2}\right)}{\sum_j \left(\frac{j}{j_{min}}\right)^{-p}} \com
\eeq
where $k = 2\pi j /L_x $, and $p$ is prescribed. Consistent with the discussion above, the eddy diffusivity at high wavenumbers are significantly more suppressed in flows with shallow kinetic energy spectrum.

 \begin{figure}[ht]
    \centering
    \includegraphics[width=.45\textwidth]{figs/beta_U_p_150.pdf}
    \includegraphics[width=.45\textwidth]{figs/beta_U_p_1024.pdf}\\
       \hspace{-.08\textwidth} \includegraphics[width=.4\textwidth]{figs/beta_U_150.pdf}
   \hspace{.03\textwidth}  \includegraphics[width=.4\textwidth]{figs/beta_U_1024.pdf}\\
    \caption{\small The parameter ratio of eddy diffusivity with mean flow relative to no-mean-flow eddy diffusivity.
    			  (Upper panels) Contours of $\beta$ as a function of the mean flow ($U$) and kinetic energy exponent ($p$),
			  with $k_{max} = 150$ (left) and $k_{max}=1024$ (right).
			  (Lower panels) A cut through the relevant exponents $p=2$ and $p=4$.}
        \label{Suppresion}
\end{figure}

 \begin{figure}[ht]
    \centering
    \includegraphics[width=.45\textwidth]{figs/betak_U_p_2.pdf}
    \includegraphics[width=.45\textwidth]{figs/betak_U_p_4.pdf}\\
       \hspace{-.08\textwidth} \includegraphics[width=.4\textwidth]{figs/betak_p_2.pdf}
   \hspace{.03\textwidth}  \includegraphics[width=.4\textwidth]{figs/betak_p_4.pdf}\\
    \caption{\small The parameter ratio of eddy diffusivity with mean flow relative to no-mean-flow eddy diffusivity.
    			  (Upper panels) Contours of $\beta$ as a function of the mean flow ($U$) and wavenumber for two kinetic
			  energy spectral exponents: $p=2$ (left) and $p=4$ (right).
			  (Lower panels) A cut through different mean flow speeds.}
        \label{Suppresion2}
\end{figure}


\section{RW revisited: spectral solutions}
Here we solve the advection-diffusion equation in a double periodic domain marching forward with fourth-order time stepper.
The main difference between this approach and the lattice model in Section 2 is that the diffusion is applied continuously instead of pulses. Indeed, an exponential time difference scheme, the diffusion is exact. As discussed in appendix A, the pulsing of the diffusion is the main source of error in the lattice model. This significantly reduces the numerical error rendering the overall solution and diagnostics more accurate (see figure \ref{KappaNum_spectral_4.}). The main downside of this approach are the problems associated with numerical stability of the time marching scheme. In particular, for the experiment discussed here, numerical stability requires about $200$ timesteps renovation cycles. 

 \begin{figure}[ht]
    \centering
    \includegraphics[width=.45\textwidth]{figs/EstKappaNump_4_spectral.pdf}
        \includegraphics[width=.45\textwidth]{figs/EstKappaNump_2_spectral.pdf}
    \caption{\small Time series of variance tendency and dissipation in a RW initial value problem advected spectrally
    and marched forward using a ETD-RK4 scheme.There is visually no difference between the two curves (Numerical diffusion
    in these solutions are rather insignificant).}
        \label{KappaNum_spectral_4}
\end{figure}

% appendices follow
\clearpage

\appendix

\section{Caveats of the RW model}

\subsection{Biases}
 This is an approximate nature of the RW model, particularly the pulsing of the diffusion, has consequences for estimating diagnostics. For instance, if one estimates the tracer dissipation after the diffusion is applied, then these estimates will be biased low. Conversely, estimating the tracer dissipation before the diffusion is applied leads to a systematic overestimation of the tracer dissipation. Breaking up the steps in smaller substeps alternating advection with diffusion, and calculating the diagnostics before and after the diffusion is applied significantly reduces these biases. Unless otherwise stated, we used the following algorithm for the simulations described here: \\


\noindent Twice
\begin{description}
\item  $\tfrac{1}{2}$ x-advection
\item   $\tfrac{1}{4}$ diffusion
\item   Diagnostics
\end{description}

\noindent  Twice
\begin{description}
\item  $\tfrac{1}{2}$ y-advection
\item   $\tfrac{1}{4}$ diffusion
\item   Diagnostics
\end{description}

We found that in some simulations there is biases related to the pulsing of diffusion significant biases may persist, particularly at relatively large $\kappa$ and shallow velocity spectrum. To understand that consider the advection velocity field $(u,v) = (0,\alpha x)$. This can thought of the y$-$advection/diffusion sub-step with 
\beq
\label{adv_alphx}
\vth_t + \alpha x \vth_y = \kappa \lap \vth\com
\eeq
where $\alpha$ is the rate of strain of the flow. Solutions to \eqref{adv_alphx} have the form
\beq
\vth(x,y,t) = A_c(t) \ee^{\ii[(k- \alpha l t)x + l y]}\com
\eeq
where
\beq
A_c(t) = A(0) \exp\{ -\kappa\left[(k^2 + l^2) t -  k l \alpha t^2 + \alpha^2 l^2 t^3/3 \right]\}\per
\eeq
Above we used the subscript c to denote that the diffusion is being applied continuously. After a
substep $\tau/2$ we have
\beq
A_c(\tau/2) = A(0) \exp\{-\kappa\left[ (k^2 + l^2) \tau/2 -  k l \alpha \tau^2/4+ \alpha^2 l^2 \tau^3/24 \right]\}\per
\eeq
If the diffusion is pulsed instead, we have
\begin{align}
A_p(\tau/2) &= A(0) \exp \{ -\kappa \left[ (k - \alpha l \tau/2)^2 + l^2  \right]\tau/2\} \nonumber \\
&= A(0) \exp\{-\kappa \left[ (k^2 + l^2) \tau/2 -  k l \alpha \tau^2/2+ \alpha^2 l^2 \tau^3/8 \right] \} \per
\end{align}
Note that pulsing the diffusion can underestimate or overestimate the solution. In particular,
\beq
\frac{A_p}{A_c} = \exp [\kappa l \alpha \tfrac{\tau^2}{2} (k - \alpha l \tau/6) ] \per
\eeq

\subsection{Numerical diffusion}
In some simulations there is a remaining bias that we account for in the form of a ``numerical diffusivity''. To estimate this numerical diffusion we consider the global variance budget in initial value problems (IVPs). Figure \ref{EstKappaNum} shows that the decay of variance in IVPs is faster than predicted by the tracer dissipation with molecular diffusivity. Numerical diffusivities  estimated by linear algebra regression are $ 1.1 \kappa$ and $1.8 \kappa$ for $p=4$ and $p=2$, respectively. Accounting for the numerical diffusivity significantly reduces biases in eddy diffusivity estimates (see figure xyz).

\begin{figure}[ht]
    \centering
    \includegraphics[width=7cm]{figs/EstKappaNump_2.pdf}
    \includegraphics[width=7cm]{figs/EstKappaNump_4.pdf}
    \caption{\small Time series of variance tendency and tracer dissipation for simulations of initial value problems with $p=2$ and $p=4$. Note that the decay of the variance decays faster than predicted by the tracer dissipation with molecular diffusivity. }
		    \label{EstKappaNum}
\end{figure}

\begin{figure}[ht]
    \centering
    \includegraphics[width=0.9\textwidth]{figs/SensKappa_p_4}
    \caption{\small Sensitivity of eddy diffusivity estimates with steep kinetic energy spectrum $p=4$ to small scale diffusivity.}
		    \label{SensKappa}
\end{figure}

\section{Derivation of Nakamura's Eddy Diffusivity}

The Nakamura effective diffusivity arises when transforming the diffusion equation into a area/contour coordinate system, i.e. A = $A(\Th,t)$, where $A$ is the area bounded by the contour $\Th$ at time $t$. Because the area bounded by a contour $\Th$ is a well-defined graph, i.e. $\Th = \Th(A,t)$ (see figure \ref{SnapAndCDF}), we have\footnote{It is very important to keep track of the difference between the tracer field $\vth$ and tracer contours defined by the \textit{label} $\vth = \Th$. It is irritating that a third of the oceanographic literature in this topic systematically confused $\vth$ with $\Th$ in their derivations. Another third randomly makes the same confusion. The last third only  occasionaly changes a $\vth$ with $\Th$.}
\beq
\Th = \vth[A(\Th,t),t]\com
\eeq
so that
\beq
\label{useful_relations}
\frac{\p \vth}{\p A}A_t + \vth_t = 0 \qquad \text{and }\qquad  \frac{\p \vth}{\p A} \frac{\p A}{\p \Th} = 1\per
\eeq

\begin{figure}[ht]
    \centering
    \includegraphics[width=8cm]{figs/snapshot_p_4.pdf}
    \includegraphics[width=7cm]{figs/fractional_area.pdf}
    \caption{\small (Left) A snapshot of tracer concentration $\vth$ from a simulation
    		of the RW model. (Right) The fractional area  of tracer concentration
		satisfying $\vth\leq\Th$, that is, the cumulative distribution function of $\vth$.}
		    \label{SnapAndCDF}
\end{figure}

\begin{figure}[ht]
    \label{SchematicsCoord}
    \centering
    \includegraphics[width=10cm]{figs/SchemaAreaCoordinates.png}
    \caption{\small Schematics of the coordinate system defined by the the tracer contour $\vth=\Th$.}
\end{figure}

The integral of the scalar function $f$ over the surface within the curves $\gamma(\Th+\dd \Th/2)$ and $\gamma(\Th-\dd \Th/2)$ (see figure \ref{SchematicsCoord}) is\footnote{We can also use the fundamental theorem of calculus  to obtain $ I_\Th(f) =\int_{\vth\leq\Th }\oint_{\gamma} f \frac{\dd l}{|\nabla \vth|} $.}

\newcommand{\A}{\p A}

\beq
\label{area_int}
\delta I_\Th(f) = \iint_{\A^+}\!\! f \, \dd x \dd y  - \iint_{\A^-}\!\! f \, \dd x \dd y  =   \delta \vth
            \oint_{\gamma} f \frac{\dd l}{|\nabla \vth|}\per
\eeq
where $\A^+ = \A(\Th+\delta \Th/2)$ and $\A^-= \A(\Th-\delta \Th/2)$.
Thus in the limit $\delta \Th \to 0$,
\beq
\frac{\p }{\p \Th}I_\Th(f) =\frac{\p}{\p \Th}  \iint_{\A}\!\! f \, \dd x \dd y   = \oint_{\gamma} f \frac{\dd l}{|\nabla \vth|}\per
\eeq
In particular, the specific case $f=1$ gives:
\beq
\label{dAdth}
\frac{\p }{\p \Th}A =  \oint_{\gamma}  \frac{\dd l}{|\nabla \vth|}\com
\eeq
where the total area of the surface bounded by $\Th$ is $A = I_\Th(1)$. The thickness average of $f$ along the $\gamma(\Th)$ is defined by
\beq
\label{defn_ave}
\langle f\rangle_\Th \defn  \frac{\oint_{\gamma} f \frac{\dd l}{|\nabla \vth|}}{\oint_{\gamma} \frac{\dd l}{|\nabla \vth|}} = \frac{\p }{\p A}I_\Th(f)\per
\eeq

With the results above, we are ready to transform the advection-diffusion equation onto a coordinate system defined by the countour $\Th$. First, we note that the rate of change of the  material surface A is given by
\beq
A_t =  \oint_\gamma \vu_c \cdot \hat{\vn} \, \dd l\com
\eeq
where the velocity of the contour, locally normal to the contour, is defined by
\beq
\vu_c = - \theta_t \frac{\nabla \theta}{|\nabla \theta|^2} = -  \frac{\theta_t}{|\nabla \theta|}\hat{\vn}\per
\eeq
Thus, using the advection-diffusion equation \eqref{ad}, we obtain
\begin{align}
\frac{\p A}{\p t} =  - \oint_\gamma \theta_t \, \frac{\dd l}{{|\nabla \theta|}} &= 
                       \oint_\gamma \, \nabla \cdot (\vu \vth) \frac{\dd l}{{|\nabla \theta|}} 
                      - \kappa \oint_\gamma \lap \vth   \frac{\dd l}{{|\nabla \theta|}} 
                      - \oint_\gamma S  \frac{\dd l}{{|\nabla \theta|}}\nonumber \\
                     & =  \frac{\p}{\p \Th}\underbrace{\iint \nabla \cdot (\vu \vth) \dd A}_{=0}
                        -  \kappa  \frac{\p}{\p \Th}\underbrace{\iint \lap \vth \dd A}_{=\oint |\nabla\vth|^2\frac{\dd l}{|\nabla \vth|}}
                        -  \frac{\p}{\p \Th}\iint S  \dd A  \per
\end{align}
Now using \eqref{useful_relations} we obtain
\begin{align}
\frac{\p \vth}{\p t} &= \kappa \frac{\p}{\p A} \iint_{\p A} \lap \vth \dd A + \frac{\p }{\p A} \iint_{\p A} S  \dd A \nonumber\\
                     &= \frac{\p}{\p A}\left(K_n(A,t) \frac{\p \vth}{\p A}\right) + \langle S \rangle_\Th\com
\end{align}
where the effective diffusivity  is
\beq
\label{KeDefn}
K_n (A, t) =  \frac{ \kappa \oint_\gamma |\nabla\vth| \dd l }{\p \vth/\p A} 
        =  \kappa  \underbrace{\frac{\p A}{\p \Th}}_{\oint_\gamma \frac{\dd l}{|\nabla \vth|}} \oint_\gamma |\nabla\vth| \dd l 
        = \frac{\p A}{\p \Th}\frac{\p }{\p A} \iint_{\p A}  \kappa  |\nabla\vth|^2 \dd x \dd y = 
        \frac{\langle \kappa  |\nabla\vth|^2 \rangle_\Th}{ (\p \vth/\p A)^2} \per
\eeq
Note that the effective diffusivity has units of (length)$^4$$\times$(time)$^{-1}$. The Nakamura eddy diffusivity is defined by
\beq
\label{knak}
\kappaN(\Th,t) = \frac{K_e}{L_{min}^2} = \frac{L_{eq}^2}{L_{min}^2}\kappa\com
\eeq
where the ``equivalent length'' is
\beq
\label{Leq}
L_{eq}^2 \defn \underbrace{\oint_\gamma \frac{\dd l}{|\nabla \vth|}}_{\defn I_2} \underbrace{\oint_\gamma |\nabla \vth| \dd l}_{\defn I_1} \com
\eeq
where the alternative expressions (useful for numerical computations) for the integrals are
\beq
I_2 = \frac{\p}{\p \Th} \iint_{\p A}  \dd x \dd y \com \qquad \text{and} \qquad I_1 = \frac{\p}{\p \Th} \iint_{\p A} |\nabla \vth|^2  \dd x \dd y\com
\eeq
Note that $\kappa I_1$ is proportional to the dissipation rate within contours.

 \begin{figure}[ht]
    \centering
    \includegraphics[width=0.45\textwidth]{figs/I1I2_p_4_nmin_2.pdf}
     \includegraphics[width=0.45\textwidth]{figs/I1I2_p_2_nmin_2.pdf}
    \caption{\small Numerical estimates of the integral $I_1$ and $I_2$ as a function of the small scale
    			diffusivity. These integrals are based only}
        \label{SensitivitykNdTh}
\end{figure}

Also in \eqref{knak}, $L_{min}$ is the minimum  (unstrained) length of the contour. In simple domains such as a periodic channel,
 $L_{min}$ is unambiguously defined as the length of the channel. In more complicated realistic geometry and source functions, the definition of $L_{min}$ may not be straightforward. In practice, one can estimate $L_{min}$ by calculating $L_{eq}$ in simulations with large Peclet number ($\kappaN \to \kappa$). 
In the oceanographic literature  $L_{eq}$ is commonly mistaken by the length of the contour (e.g., Ferrari and Nikurashin, 2010). This is only true if $|\nabla \vth |$ is constant, and represents a lower bound on $L_{eq}$, obtained using the Cauchy-Schwartz inequality\footnote{$\oint_\gamma f^2 \dd l \oint_\gamma g^2 \dd l \ge \left(\oint_\gamma f\,g \,\dd l\right)^2.$}:

\beq
L_{eq}^2 \ge L^2 = \Big(\oint_\gamma \dd l \Big)^2\per
\eeq
While the equivalent length notation above provides a useful interpretation, in practice  one does not compute
line integrals. Instead, the third equality in \eqref{KeDefn} suggests a straightforward numerical approximation
\beq
K_e(\Th,t) \approx \frac{\delta I_\Th(1)}{(\delta \Th)^2} \delta I_\Th(\kappa |\nabla \vth|^2) \com 
\eeq
where $\delta I_\theta(f)$ is the approximate area integral of $f$ within the contours $\Th + \dd \Th/2$ and $\Th - \dd \Th/2$ (see figure 5). These approximate integrals are computed simply by counting pixels:
\beq
\delta I_\Th(f) = \sum_j f \dd x \dd y\com \qquad \Th - \dd \Th/2 \leq \vth \leq \Th + \dd \Th/2 \com
\eeq
 where the summation on $j$ represents the pixels within the contour. In practice the diagnostic is not very sensitive to the 
 contour spacing $\dd \Th$.
 
 Finally note that the Nakamura diffusivity is a function of the the contour $\Th$ (or equivalently of the area bounded by the contour). With simple geometry of initial conditions and source function, $\kappa_N(\Th,t)$  can be mapped to physical space, e.g. as direction of the tracer gradient or the equivalent latitude. 
  
  
 \begin{figure}[ht]
    \centering
    \includegraphics[width=7.5cm]{figs/sensitivity_kappaN_dtheta.pdf}
        \includegraphics[width=7.5cm]{figs/sensitivity_kappaN_mean_dtheta.pdf}\\
    \caption{\small Sensitivity of the Nakamura diffusivity estimate to the contour spacing $\dd \Th$. In this example, the Nakamura eddy diffusivity was computed for a single snapshot in Figure \ref{SnapAndCDF}a.}
        \label{SensitivitykNdTh}
\end{figure}
  
  
 
%\section{A more direct derivation}
%Just average \eqref{ad} along the contour $\Th$ (i.e., using definition \eqref{defn_ave}) to obtain
%
%\beq
%\underbrace{\frac{\p}{\p A} \iint \vth_t \dd A}_{=\frac{\p \Th}{\p A}\frac{\p}{\p \Th} \iint \vth_t \dd A = \frac{\p \Th}{\p A} \oint \vth_t \dd A} =  \frac{\p}{\p A}\left(K_{e}(A,t) \frac{\p \Th}{\p A}\right) + \langle S \rangle_\Th  \per
%\eeq
%Because 
%\beq
%\frac{\p A}{\p t} =  - \oint \theta_t \, \frac{\dd l}{{|\nabla \theta|}}\com
%\eeq
%we obtain, using \eqref{ThtoA},
%\beq
%\label{ad_area}
%\frac{\p \Th}{\p t} = \frac{\p}{\p A}\left(K_{e}(A,t) \frac{\p \Th}{\p A}\right) + \langle S \rangle_\Th\per
%\eeq


\end{document}
